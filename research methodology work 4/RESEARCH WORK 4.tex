\documentclass[7pt]{article}
\raggedright
\parindent=0in \parskip=8pt
\usepackage{graphicx}
\usepackage[margin=1in]{geometry} % 1 inch margins all around
\begin{document}

\begin{Huge}
\begin{center}
\begin{normalsize}

\textbf{FACULTY OF COMPUTING AND INFORMATICS TECHNOLOGY} \\
\textbf{SCHOOL OF COMPUTING AND INFORMATICS TECHNOLOGY} \\
\textbf{DEPARTMENT OF COMPUTER SCIENCE} \\
\textbf{BACHELOR OF SCIENCE IN COMPUTER SCIENCE} \\
\textbf{YEAR 2} \\
\textbf{BIT 2207 RESEARCH METHODOLOGY} \\
\textbf{Course Work: Assignment 4}\\
\end{normalsize}
\end{center}
\end{Huge}

\begin{center}
\begin{tabular}{l l l}
\textbf{NAME}  & \textbf{REGISTRATION NUMBER} & \textbf{STUDENT NUMBER} \\
SSENTOOGO CHARLES & 16/U/11715/EVE & 216014457 \\
\end{tabular}

\paragraph{•}
\textbf{Lecturer}: Mr. Earnest Mwebaze
\end{center}

\newpage

A LITERATURE REVIEW ABOUT ANDROID STUDIO
\section{Introduction}
Android Studio is the official integrated development environment (IDE) for Google's Android operating system, built on JetBrains' IntelliJ IDEA software and designed specifically for Android development.\cite{r1} It is available for download on Windows, macOS and Linux based operating systems. It is a replacement for the Eclipse Android Development Tools (ADT) as primary IDE for native Android application development.
\\
Android Studio was announced on May 16, 2013 at the Google I/O conference. It was in early access preview stage starting from version 0.1 in May 2013, then entered beta stage starting from version 0.8 which was released in June 2014. The first stable build was released in December 2014, starting from version 1.0. The current stable version is 3.0 released in October 2017.




\section{Services }
\subsection{Adroid Studio API}
 For developers, the Android 4.0.3 platform is available as a downloadable component for the Android SDK. The downloadable platform includes an Android library and system image, as well as a set of emulator skins and more. To get started developing or testing against Android 4.0.3, use the Android SDK Manager to download the platform into your SDK.

\subsection{Android Studio Requirements For Windows OS}
Microsoft Windows 7/8/10 (32-bit or 64-bit)
2 GB RAM minimum, 8 GB RAM recommended.
2 GB of available disk space minimum, 4 GB Recommended (500 MB for IDE + 1.5 GB for Android SDK and emulator system image)
1280 x 800 minimum screen resolution.
JDK 8\cite{r2}.
\section{Veiws from users}
At present, more than 76.6 percent of the Smartphone’s, including HTC, LG and Samsung Models use Android as their operating system (OS), and expecting that Android will be in smart watches, laptops, car very soon. Android powered devices including tablets have become the foremost need of all the tech-savvy people across the world and the prime reason is it provides an open source platform for the development of great apps plus allows app developers to immediately publish them. Instead lots of developers want to get associated with Android application because of incredible growth.
\section{conclusion}

Android Studio is now the only official development tool that will support the full-featured Android SDKs in the future. This is about the right time to move towards Android Studio if you would like to continue to enjoy new upcoming Android features offered by Google. The reason is simple: Google will no longer provide updates for the Eclipse plug-in or ADT (Android Development Toolkit) and will only release new SDKs for its own Android Studio.

\begin{thebibliography}{9}
\bibitem{r1}Xavier Ducrot \textit{An IDE built for Android}
Internet:https://android-developers.googleblog.com/2013/05/android-studio-ide-built-for-android.html  15 May 2013 [10 march 2018]\bibitem{r2}Jeff Friesen \textit{Android studio For Begginers}
Internet:https://www.javaworld.com/article/3095406/android/android-studio-for-beginners-part-1-installation-and-setup.html  aug 23,2016[10 march 2018]
\end{thebibliography}

\end{document}
